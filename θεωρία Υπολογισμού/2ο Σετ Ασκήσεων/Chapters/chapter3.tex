\section{Μη ντετερμινισμός και κανονικότητα αυτομάτων}

\begin{figure}[h]
	\centering
	\begin{tikzpicture}[>=stealth',shorten >=1pt,auto, node distance=2.3 cm, scale = 1, transform shape]
		\node[initial,state, thick]  			 (q1)   						{$q_1$};
		\node[state, thick, accepting]     (q2) [right of=q1]   {$q_2$};
		\node[state, thick]           				(q3) [right of=q2]   {$q_3$};
		\node[state, thick]           				(q4) [right of=q3]   {$q_4$};
		
		\path[->] (q1) edge [thick]     							 node [align=center]  {$b$} (q2)
		(q1) edge [thick, bend right=-40 ]    node [align=center]  {$e$} (q3)
		(q2) edge [thick, bend right=-10 ]    node [align=center]  {$a$} (q3)
		(q3) edge [thick, bend right=-10 ]    node [align=center]  {$b$} (q2)
		(q3) edge [thick]     							   node [align=center]  {$a$} (q4)
		(q3) edge [thick,loop above]   		     node [align=center]  {$a$} (q3)
		(q4) edge [thick, bend right=-25]     node [align=center]  {$e$} (q1);
	\end{tikzpicture}
	\caption{Μη ντετερμινιστικό αυτόματο Μ}
	\label{fig:not_det}
\end{figure}

\subsection[Ερώτημα α]{\textbf{α) }}Κατασκευάστε αναλυτικά ένα ισοδύναμο ντετερμινιστικό αυτόματο $Μ'$

\noindent\\
Για την κατασκευή του ισοδύναμου ντετερμινιστικού αυτόματου, αρχικά είναι απαραίτητο να υπολογιστούν τα $\Epsilon$ κάθε κατάστασης. Στο $\Epsilon$ μίας κατάστασης τοποθετούνται όλες οι καταστάσεις προς τις οποίες υπάρχει κενή μετάβαση, ενώ είναι προφανές πως πάντα θα περιέχεται η ίδια η κατάσταση που εξετάζεται.
\begin{align*}
	&\Epsilon (q_1) = \{q_1, q_3\}\\
	&\Epsilon (q_2) = \{q_2\} \\
	&\Epsilon (q_3) = \{q_3\}\\
	&\Epsilon (q_4) = \{q_1, q_4\}
\end{align*}

\noindent\\
Έχοντας κατασκευάσει τα $\Epsilon$, κατασκευάζεται η νέα αρχική κατάσταση η οποία ισούται με το $\Epsilon$ της προηγούμενης αρχικής κατάστασης, δηλαδή της $q_1$:
\begin{figure}[h]
	\centering
	\begin{tikzpicture}[>=stealth',shorten >=1pt,auto, scale = 1, transform shape]
		\node[initial,state, thick, shape=ellipse]	 (q0)	 																	  {$\{q_1, q_3\}$};
	\end{tikzpicture}
\end{figure}

\noindent\\
Στην συνέχεια, εξετάζονται όλες οι πιθανές μεταβάσεις από την αρχική κατάσταση:
\begin{equation*}
	\begin{rcases*}
		q_1 \xrightarrow{a} X\\
		q_3 \xrightarrow{a} $\Epsilon$ (q_3) \bigcup $\Epsilon$ (q_4)
	\end{rcases*} \text{Άρα η μετάβαση a οδηγεί στο } \Epsilon (q_3) \cup \Epsilon (q_4)\\
\end{equation*}
\begin{equation}
	\begin{rcases*}
		q_1 \xrightarrow{b} $\Epsilon$ (q_2)\\
		q_3 \xrightarrow{b} $\Epsilon$ (q_2)
	\end{rcases*} \text{Άρα η μετάβαση b οδηγεί στο } \Epsilon (q_2)
\end{equation}

\noindent
οπότε, το σχήμα μετασχηματίζεται ως εξής:
\begin{figure}[h]
	\centering
	\begin{tikzpicture}[>=stealth',shorten >=1pt,auto, scale = 1, transform shape]
		\node[initial,state, thick, shape=ellipse]	 (q0)	 																	  {$\{q_1, q_3\}$};
		\node[state, thick, shape=ellipse]									 (q1)    [above right=0.5cm and 1cm of q0]     {$\{q_1, q_3, q_4\}$};
		\node[state, thick, accepting]  				   						(q2)    [below right=0.2cm and 1.6cm of  q0]  {$\{q_2\}$};
		
		\path[->] (q0) edge [thick]     				   		 	node [align=center]  {$a$} (q1)
		(q0) edge [thick, below]     				 node [align=center, xshift=-7pt, yshift=-1pt]  {$b$} (q2);
	\end{tikzpicture}
\end{figure}

\clearpage
\noindent
Ομοίως, εξετάζονται όλες οι πιθανές μεταβάσεις από την κατάσταση $\{q_1, q_3, q_4\}$
\begin{equation*}
	\begin{rcases*}
		q_1 \xrightarrow{a} X\\
		q_3 \xrightarrow{a} $\Epsilon$ (q_3) \bigcup $\Epsilon$ (q_4)\\
		q_4 \xrightarrow{a} X
	\end{rcases*} \text{Άρα η μετάβαση a οδηγεί στο } \Epsilon (q_3) \cup \Epsilon (q_4)\\
\end{equation*}
\begin{equation*}
	\begin{rcases*}
		q_1 \xrightarrow{b} $\Epsilon$ (q_2)\\
		q_3 \xrightarrow{b} $\Epsilon$ (q_2)\\
		q_4 \xrightarrow{b} X
	\end{rcases*} \text{Άρα η μετάβαση b οδηγεί στο } \Epsilon (q_2)
\end{equation*}

οπότε, το σχήμα μετασχηματίζεται ως εξής:
\begin{figure}[h]
	\centering
	\begin{tikzpicture}[>=stealth',shorten >=1pt,auto, scale = 1, transform shape]
		\node[initial,state, thick, shape=ellipse]	 (q0)	 																	  {$\{q_1, q_3\}$};
		\node[state, thick, shape=ellipse]									 (q1)    [above right=0.5cm and 1cm of q0]     {$\{q_1, q_3, q_4\}$};
		\node[state, thick, accepting]  				   						(q2)    [below right=0.2cm and 1.6cm of  q0]  {$\{q_2\}$};
		
		\path[->] (q0) edge [thick]     				   		 	node [align=center]  {$a$} (q1)
		(q0) edge [thick, below]     				 node [align=center, xshift=-7pt, yshift=-1pt]  {$b$} (q2)
		(q1) edge [thick,loop above]    		 node [align=center]  {$a$} (q1)
		(q1) edge [thick]    			   		  		   node [align=center]  {$b$} (q2);
	\end{tikzpicture}
\end{figure}
\noindent\\\\
Ακολουθώντας την ίδια διαδικασία για την κατάσταση $\{q_2\}$
\begin{equation*}
	\begin{rcases*}
		q_2 \xrightarrow{a} $\Epsilon$ (q_3)
	\end{rcases*} \text{Άρα η μετάβαση a οδηγεί στο } \Epsilon (q_3)\\
\end{equation*}
\begin{equation*}
	\begin{rcases*}
		q_2 \xrightarrow{b} X
	\end{rcases*} \text{Άρα η μετάβαση b οδηγεί σε καταβόθρα}
\end{equation*}
\begin{figure}[h]
	\centering
	\begin{tikzpicture}[>=stealth',shorten >=1pt,auto, scale = 1, transform shape]
		\node[initial,state, thick, shape=ellipse]	 (q0)	 																	  {$\{q_1, q_3\}$};
		\node[state, thick, shape=ellipse]									 (q1)    [above right=0.5cm and 1cm of q0]     {$\{q_1, q_3, q_4\}$};
		\node[state, thick]           							    					(q3) 	[right=2cm  of q1]     			   				  	  {$\{q_3\}$};
		\node[state, thick, accepting]  				   						(q2)    [below right=0.2cm and 1.6cm of  q0]  {$\{q_2\}$};
		\node[state, thick]										    					(q4) 	[right= 3cm of q2]  			   			 		 {$\{ \}$};
		
		\path[->] (q0) edge [thick]     				   		 	node [align=center]  {$a$} (q1)
		(q0) edge [thick, below]     				 node [align=center, xshift=-7pt, yshift=-1pt]  {$b$} (q2)
		(q1) edge [thick,loop above]    		 node [align=center]  {$a$} (q1)
		(q1) edge [thick]    			   		  		   node [align=center]  {$b$} (q2)
		(q2) edge [thick, bend right=-10 ]    node [align=center]  {$a$} (q3)
		(q2) edge [thick, below]    				 node [align=center]  {$b$} (q4)
		(q4) edge [thick, loop above]   		 node [align=center]  {$a,b$} (q4);
	\end{tikzpicture}
\end{figure}
\noindent
Ακολουθώντας την ίδια διαδικασία για την κατάσταση $\{q_3\}$
\begin{equation*}
	\begin{rcases*}
		q_3 \xrightarrow{a} $\Epsilon$ (q_3) \bigcup $\Epsilon$ (q_4)
	\end{rcases*} \text{Άρα η μετάβαση a οδηγεί στο } \Epsilon (q_3) \bigcup \Epsilon (q_4)\\
\end{equation*}
\begin{equation*}
	\begin{rcases*}
		q_3 \xrightarrow{b} $\Epsilon$ (q_2)
	\end{rcases*} \text{Άρα η μετάβαση b οδηγεί στο } \Epsilon (q_2)
\end{equation*}

οπότε το τελικό σχήμα προκύπτει ως εξής:
\clearpage
\begin{figure}[h]
	\centering
	\begin{tikzpicture}[>=stealth',shorten >=1pt,auto, scale = 1, transform shape]
		\node[initial,state, thick, shape=ellipse]	 					(q0)	 																	  {$\{q_1, q_3\}$};
		\node[state, thick, shape=ellipse]									 (q1)    [above right=0.5cm and 1cm of q0]     {$\{q_1, q_3, q_4\}$};
		\node[state, thick]           							    					(q3) 	[right=2cm  of q1]     			   				  	  {$\{q_3\}$};
		\node[state, thick, accepting]  				   						(q2)    [below right=0.2cm and 1.6cm of  q0]  {$\{q_2\}$};
		\node[state, thick]										    					(q4) 	[right= 3cm of q2]  			   			 		 {$\{ \}$};
		
		\path[->] (q0) edge [thick]     				   		 	node [align=center]  {$a$} (q1)
		(q0) edge [thick, below]     				 node [align=center, xshift=-7pt, yshift=-1pt]  {$b$} (q2)
		(q1) edge [thick,loop above]    		 node [align=center]  {$a$} (q1)
		(q1) edge [thick]    			   		  		   node [align=center]  {$b$} (q2)
		(q3) edge [thick, above]    				 node [align=center]  {$a$} (q1)
		(q3) edge [thick, bend right=-10 ]    node [align=center]  {$b$} (q2)
		(q2) edge [thick, bend right=-10 ]    node [align=center]  {$a$} (q3)
		(q2) edge [thick, below]    				 node [align=center]  {$b$} (q4)
		(q4) edge [thick, loop above]   		 node [align=center]  {$a,b$} (q4);
	\end{tikzpicture}
\end{figure}
\noindent\\
Τελική κατάσταση θεωρείται μόνο η κατάσταση που περιέχει την κατάσταση $q_2$\\\\

\subsection[Ερώτημα β]{\textbf{β) }} Κατασκευάστε αναλυτικά την κανονική έκφραση για την L(M??) με σειρά απαλοιφής q4, q2, q1, q3.

\noindent\\
Για την απαλοιφή μίας κατάστασης, είναι απαραίτητο να αντικατασταθούν όλες οι μεταβάσεις που αξιοποιούν την εκάστοτε κατάσταση, είτε δημιουργώντας νέες είτε ανανεώνοντας τις ήδη υπάρχουσες.

\noindent\\
Για την μετατροπή του μη ντετερμινιστικού αυτόματου (figure \ref{fig:not_det}) απαιτείται η εισαγωγή 2 επιπλέον καταστάσεων οι οποίες αντικαθιστούν την υπάρχουσα αρχική και την υπάρχουσα τελική κατάσταση. Μεταξύ των νέων και των παλιών καταστάσεων δημιουργείται κενή μετάβαση.
%%%%%%%%%%%%%%%%%%%%%%%%%%%%%%%% Init state %%%%%%%%%%%%%%%%%%%%%%%%%%%%%%%%
\begin{center}
	\begin{tikzpicture}[>=stealth',shorten >=1pt,auto, node distance=2.3 cm, scale = 1, transform shape]
		\node[initial,state, thick]  			 (q0) [color=orange]  							{$q_0$};
		\node[state, thick]  			 			(q1) [right of=q0]   							   {$q_1$};
		\node[state, thick]     					(q2) [right of=q1]   							  {$q_2$};
		\node[state, thick]           				(q3) [right of=q2]   							   {$q_3$};
		\node[state, thick]           				(q4) [right of=q3]   							   {$q_4$};
		\node[state, thick, accepting]     (q5) [below of=q2, color=orange]   	{$q_5$};
		
		\path[->] 	 (q0) edge [thick, color=orange]      node [align=center]  {$e$} (q1)
		(q1) edge [thick]     							 node [align=center]  {$b$} (q2)
		(q1) edge [thick, bend right=-40 ]    node [align=center]  {$e$} (q3)
		(q2) edge [thick, bend right=-10 ]    node [align=center]  {$a$} (q3)
		(q3) edge [thick, bend right=-10 ]    node [align=center]  {$b$} (q2)
		(q3) edge [thick]     							   node [align=center]  {$a$} (q4)
		(q3) edge [thick,loop above]   		     node [align=center]  {$a$} (q3)
		(q4) edge [thick, bend right=-25]     node [align=center]  {$e$} (q1)
		(q2) edge [thick, color=orange]       node [align=center]  {$e$} (q5);
	\end{tikzpicture}
\end{center}

\noindent
\textbf{Απαλοιφή της κατάστασης $q_4$}
\noindent\\Είναι απαραίτητο να αντικατασταθούν οι εξής μεταβάσεις:
η μετάβαση $q_3 \xrightarrow{q_4} q_1$. Η αντικατάσταση γίνεται με δημιουργία νέας μετάβασης από την κατάσταση $q_3$ στην κατάσταση $q_4$ όταν το (σύμβολο???) είναι a:
%%%%%%%%%%%%%%%%%%%%%%%%%%%%%%% Eliminate q4 %%%%%%%%%%%%%%%%%%%%%%%%%%%%%%%
\begin{center}
	\begin{tikzpicture}[>=stealth',shorten >=1pt,auto, node distance=2.3 cm, scale = 1, transform shape]
		\node[initial,state, thick]  			 (q0) [color=orange]  							{$q_0$};
		\node[state, thick]  			 			(q1) [right of=q0]   							   {$q_1$};
		\node[state, thick]     					(q2) [right of=q1]   							  {$q_2$};
		\node[state, thick]           				(q3) [right of=q2]   							   {$q_3$};
		\node[state, thick, accepting]     (q5) [below of=q2, color=orange]   	{$q_5$};
		
		
		\path[->] 	 (q0) edge [thick, color=orange]      						 node [align=center]  									   {$e$} (q1)
		(q1) edge [thick]     							 						node [align=center]  									  {$b$} (q2)
		(q1) edge [thick, bend right=-40]   					   node [align=center]  									 {$e$} (q3)
		(q2) edge [thick, bend right=-10]   					   node [align=center] 									 	 {$a$} (q3)
		(q3) edge [thick, bend right=-10]   					   node [align=center]  									 {$b$} (q2)
		(q3) edge [thick,loop above]   		   						  node [align=center]  										{$a$} (q3)
		(q3) edge [thick, color=red, , bend right=-37]     node [align=center, xshift=-10pt, above]  {$a$} (q1)
		(q2) edge [thick, color=orange]       						node [align=center]  									  {$e$} (q5);
	\end{tikzpicture}
\end{center}

\clearpage
\noindent
\textbf{Απαλοιφή της κατάστασης $q_2$}
\noindent\\Είναι απαραίτητο να αντικατασταθούν οι εξής μεταβάσεις:
\begin{align*}
	q_1 \xrightarrow{q_2} q_3 &\xRightarrow{} ba\\
	q_1 \xrightarrow{q_2} q_5 &\xRightarrow{} b\\
	q_3 \xrightarrow{q_2} q_3 &\xRightarrow{} ba\\
	q_3 \xrightarrow{q_2} q_5 &\xRightarrow{} b\\
\end{align*}
\noindent
Η πρώτη μετάβαση προστίθεται στην ήδη υπάρχουσα κενή μετάβαση που υπήρχε μεταξύ της κατάστασης $q_1$ και της $q_3$ ενώ η δεύτερη μετάβαση δημιουργείται καθώς δεν προϋπάρχει μετάβαση. Αντίστοιχα, η τρίτη μετάβαση προστίθεται στο ήδη υπάρχον self loop της κατάστασης $q_3$ ενώ για την τέταρτη μετάβαση δημιουργεί και πάλι νέα.
%%%%%%%%%%%%%%%%%%%%%%%%%%%%%%% Eliminate q2 %%%%%%%%%%%%%%%%%%%%%%%%%%%%%%%
\begin{center}
	\begin{tikzpicture}[>=stealth',shorten >=1pt,auto, node distance=2.3 cm, scale = 1, transform shape]
		\node[initial,state, thick]  			 (q0) [color=orange]  																		 	 {$q_0$};
		\node[state, thick]  			 			(q1) [right of=q0]   							   												   {$q_1$};
		\node[state, thick]           				(q3) [right=3cm of q1]   					  												  {$q_3$};
		\node[state, thick, accepting]      (q5) [below right = 0.9cm and -2.5cm of q3, color=orange]   	{$q_5$};
		
		
		\path[->] 	 (q0) edge [thick, color=orange]      						 node [align=center]  						{$e$} (q1)
		(q1) edge [thick, color=red, below]        				  node [align=center, xshift=-5pt] 	{$b$} (q5)
		(q1) edge [thick, color=red, bend right=-15]   	   node [align=center]  					  {$e \cup ba$} (q3)
		(q3) edge [thick,loop above]   		   						  node [align=center]  						 {$a$\color{red}{$\cup ba$}} (q3)
		(q3) edge [thick, color=red]       							    node [align=center]  					  {$b$} (q5)
		(q3) edge [thick, bend right=-15]     					   node [align=center, above]  			{$a$} (q1);
	\end{tikzpicture}
\end{center}

\noindent\\
\textbf{Απαλοιφή της κατάστασης $q_1$}
\noindent\\Ομοίως, είναι απαραίτητο να αντικατασταθούν οι εξής μεταβάσεις:
\begin{align*}
	q_0 \xrightarrow{q_1} q_3 &\xRightarrow{} ba\\
	q_0 \xrightarrow{q_1} q_5 &\xRightarrow{} b\\
	q_3 \xrightarrow{q_1} q_3 &\xRightarrow{}a (e \cup ba)\\
	q_3 \xrightarrow{q_1} q_4&\xRightarrow{} ab\\
\end{align*}
\noindent
Για την πρώτη μετάβαση δημιουργείται νέα μετάβαση μεταξύ της $q_0$ και $q_3$ ενώ νέα μετάβαση δημιουργείται και για την δεύτερη μετάβαση μεταξύ της $q_0$ και $q_5$. Αντίστοιχα, η τρίτη μετάβαση προστίθεται στο ήδη υπάρχον self loop της κατάστασης $q_3$ ενώ το ίδιο γίνεται και για την τέταρτη μετάβαση.
%%%%%%%%%%%%%%%%%%%%%%%%%%%%%%% Eliminate q1 %%%%%%%%%%%%%%%%%%%%%%%%%%%%%%%
\begin{center}
	\begin{tikzpicture}[>=stealth',shorten >=1pt,auto, node distance=2.3 cm, scale = 1, transform shape]
		\node[initial,state, thick]  			 (q0) [color=orange]  							 												 {$q_0$};
		\node[state, thick]           				(q3) [right=3cm of q0]   					  												  {$q_3$};
		\node[state, thick, accepting]      (q5) [below right = 0.9cm and -2.5cm of q3, color=orange]   	{$q_5$};
		
		
		\path[->] 	 (q0) edge [thick, color=red]      node [align=center]     {$e \cup ba$} (q3)
		(q0) edge [thick, color=red]      node [align=center]  	  {$b$} (q5)
		(q3) edge [thick,loop above]    node [align=center]  	{$(a\cup ba)$\color{red}{$\cup(a (e \cup ba))$}} (q3)
		(q3) edge [thick]       				  node [align=center]  	  {$b$\color{red}{$\cup ab$}} (q5);
	\end{tikzpicture}
\end{center}

\clearpage
\noindent
\textbf{Απαλοιφή της κατάστασης $q_3$}
\noindent\\ Τέλος, είναι απαραίτητο να αντικατασταθεί η μετάβαση $q_0 \xrightarrow{q_3} q_4$.  Η αντικατάσταση γίνεται προσθέτοντας την εξής έκφραση στην ήδη υπάρχουσα μετάβαση μεταξύ $q_0$ και $q_5$:
\begin{equation}
	\bigg(e \cup ba\bigg) \bigg((a\cup ba)\cup \big(a (e \cup ba) \big) \bigg)^* \bigg(b\cup ab \bigg)
\end{equation}

\noindent\\
Στην παραπάνω έκφραση ο πρώτος όρος προέρχεται από την μετάβαση μεταξύ της $q_0$ και $q_3$, ο δεύτερος αποτελεί το self loop στην $q_3$, για αυτό και χρησιμοποιήθηκε αστερίσκος και ο τρίτος όρος προέρχεται από την μετάβαση μεταξύ της $q_3$ και $q_5$.
%%%%%%%%%%%%%%%%%%%%%%%%%%%%%%% Eliminate q3 %%%%%%%%%%%%%%%%%%%%%%%%%%%%%%%
\begin{center}
	\begin{tikzpicture}[>=stealth',shorten >=1pt,auto, node distance=2.3 cm, scale = 1, transform shape]
		\node[initial,state, thick]  			 (q0) [color=orange]  							 			{$q_0$};
		\node[state, thick, accepting]     (q5) [right = 9cm of q0, color=orange]   	{$q_5$};
		
		\path[->]  (q0) edge [thick]	node [align=center]		{\textcolor{red}{$\Bigg( \bigg(e \cup ba\bigg) \bigg((a\cup ba)\cup \big(a (e \cup ba) \big) \bigg)^* \bigg(b\cup ab \bigg)\Bigg) \bigcup$} $b$} (q5);
	\end{tikzpicture}
\end{center}

Έτσι, η τελική κανονική έκφραση είναι η εξής : $\Bigg( \bigg(e \cup ba\bigg) \bigg((a\cup ba)\cup \big(a (e \cup ba) \big) \bigg)^*b\cup ab\Bigg) \bigcup b$