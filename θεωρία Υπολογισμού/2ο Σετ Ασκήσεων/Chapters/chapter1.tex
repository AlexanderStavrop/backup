\section{Γραμματικές Χωρίς Συμφραζόμενα}

\subsection{ $L_1 = \{1^n0^m1^k : m > n+ k\}$}
\noindent\\
\begin{minipage}{0.4\textwidth}
\noindent\\
Γραμματική αναπαράσταση της γλώσσας 
\begin{align*}
	G = (V, Σ, &R, S)\\
			V  = \{&0, 1, S, S_1, S_2\}\\
			Σ = \{&0, 1\}\\
			R = \{&S \xrightarrow{} S_1 0 S_2, \\
					 &S_1 \xrightarrow{} 1 S_1 0  \\ 
					 &S_1 \xrightarrow{} S_1 0     \\ 
					 &S_1 \xrightarrow{} e            \\
					 &S_2 \xrightarrow{} 0 S_2 1  \\
					 &S_2 \xrightarrow{} 0S_2      \\
					 &S_2 \xrightarrow{} e \}
\end{align*}
\end{minipage}
\noindent
\begin{minipage}{0.6\textwidth}
\noindent\\
Συντακτικό δέντρο επίλυσης της συμβολοσειράς: 1000000111\\ \\
\begin{tikzpicture}[  level 1/.style={level distance=12mm,sibling distance=28mm},
										level 2/.style={level distance=10mm,sibling distance=21mm},
										level 3/.style={level distance=10mm, sibling distance=14mm},
										level 4/.style={level distance=10mm, sibling distance=7mm},
										font=\scriptsize,inner sep=1pt,every node/.style={minimum size=3ex}, text=black, >=latex ]
	\node (1){\(S\)} 
	child { node (11)  {$S_1$}
		child { node (21) {\textcolor{red}{1}}}
		child { node (22) {$S_1$}
			child { node (31) {e}}
		}
		child { node (23) {\textcolor{red}{0}}}
	}
	child { node (12) {\textcolor{red}{0}}}
	child { node (13) {$S_2$}
		child { node (23) {\textcolor{red}{0}}} 
		child { node (24) {$S_2$}
			child { node (32) {\textcolor{red}{0}}} 
			child { node (33) {$S_2$}
				child { node (41) {\textcolor{red}{0}}} 
				child { node (42) {$S_2$}
					child { node (51) {e}}
				}
				child { node (43) {\textcolor{red}{1}}}
			}
			child { node (34) {\textcolor{red}{1}}}
		}
		child { node (25) {\textcolor{red}{1}}}
	};
\end{tikzpicture}
\end{minipage}

\noindent\\
\subsection{$L_2 = \{w \in {a, b}^* :$ το πλήθος των β είναι 3k + 2, όπου k το πλήθος των a\}}
\label{sub12}
\noindent\\
\begin{minipage}{0.4\textwidth}
	\noindent\\
	Γραμματική αναπαράσταση της γλώσσας 
	\begin{align*}
	G = (V, Σ, &R, S)\\
	V  = \{&a, b, S, S_1\}\\
	Σ = \{&a, b\}\\
	R = \{&S \xrightarrow{} S_1 b S_1 b S_1\\
			 &S_1 \xrightarrow{} aS_1bbb S_1\\
			 &S_1 \xrightarrow{} baS_1bb S_1\\
			 &S_1 \xrightarrow{} bbaS_1b S_1\\
			 &S_1 \xrightarrow{} bbba S_1\\
			 &S_1 \xrightarrow{} e \}
\end{align*}
\end{minipage}
\noindent
\begin{minipage}{0.6\textwidth}
	Συντακτικό δέντρο επίλυσης της συμβολοσειράς: babbbbabbb\\\\
	\begin{tikzpicture}[  level 1/.style={level distance=15mm,sibling distance=25mm},
		level 2/.style={level distance=15mm,sibling distance=6mm},
		level 3/.style={level distance=14mm, sibling distance=2mm},
		level 4/.style={level distance=14mm, sibling distance=7mm},
		font=\scriptsize,inner sep=1pt,every node/.style={minimum size=3ex}, text=black, >=latex ]
		
		\node (1){\(S\)} 
		child { node (11)  {$S_1$}
			child { node (21)  {e}}
		}
		child { node (12)  {\textcolor{red}{b}}}
		child { node (13)  {$S_1$}	
			child { node (22)  {\textcolor{red}{a}}}
			child { node (23)  {$S_1$}
				child { node (31)  {e}}
			}
			child { node (24)  {\textcolor{red}{b}}}
			child { node (25)  {\textcolor{red}{b}}}
			child { node (25)  {\textcolor{red}{b}}}
			child { node (23)  {$S_1$}
				child { node (32)  {e}}
			}
		}
		child { node (14)  {\textcolor{red}{b}}}
		child { node (15)  {$S_1$}	
			child { node (22)  {\textcolor{red}{a}}}
			child { node (23)  {$S_1$}
				child { node (31)  {e}}
			}
			child { node (24)  {\textcolor{red}{b}}}
			child { node (25)  {\textcolor{red}{b}}}
			child { node (25)  {\textcolor{red}{b}}}
			child { node (23)  {$S_1$}
				child { node (32)  {e}}
			}
		};
	\end{tikzpicture}
\end{minipage}