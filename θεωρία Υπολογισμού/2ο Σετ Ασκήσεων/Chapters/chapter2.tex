\section{Γραμματικές Χωρίς Συμφραζόμενα}

\subsection{$L_1 = \{uw : u \in{a, c}^*, w \in a^*b^*a^* \text{ και } 2|u| = |w| \}$}

\noindent\\
Εφόσον, δεν είναι γνωστό το μήκος του input αλλά μόνο το σύμβολο στην αρχή του και το σύμβολο στην κορυφή της στοίβας, η γραμματική σχεδιάστηκε ώστε να τρέχουν παράλληλα διαδικασίες έως ότου τερματίσουν.
\begin{align*}
M = (Κ, Σ,& Γ, Δ, s, F)\\
			K = \{&s, q, d, f\}\\
			Σ = \{&a, b, c\}\\
			Γ = \{&0, 1\}\\
			F  = \{&f\}\\
			Δ = \{& Δ_1 : \big(  (s, e, e), (q, 0) \big)\\
				  	 & Δ_2 : \big(  (q, a, e), (q, 11) \big)\\
				 	 & Δ_3 : \big(  (q, c, e), (q, 11) \big)\\
				 	 & Δ_4 : \big(  (q, a, 1), (d, e) \big)\\
				 	 & Δ_5 : \big(  (q, b, 1), (d, e) \big)\\
				 	 & Δ_6 : \big(  (d, a, 1), (d, e) \big)\\
				 	 & Δ_7 : \big(  (d, b, 1), (d, e) \big)\\
					 & Δ_8 : \big(  (d, e, 0), (f, e) \big)\}
\end{align*}

\noindent\\\\
Υπολογισμός αποδοχής συμβολοσειράς acaaabbba σύμφωνα με το παραπάνω αυτόματο. 
\begin{align*}
	(s, aca&aabbba, e)  \vdash^{Δ_1} (q, acaaabbba, 0)  \vdash^{Δ_2} (q, caaabbba, 110) \vdash^{Δ_3} (q, aaabbba, 11110) \xRightarrow{}\\ 
					&\hspace{1.7cm} \xRightarrow{}	\begin{cases}
											 							    		\vdash^{Δ2} (q, aabbba, 1111110) 	\begin{cases}
																													 							    			\vdash^{Δ2} 						(q, abbba, 111111110) 	 \text{ (1)} \\
																													 							    			\vdash^{Δ4} 					(d, abbba, 111110) \text{ (2)} 
																													 							    		\end{cases} \\
													 							    \vdash^{Δ4} (d, aabbba, 1110) \vdash^{Δ_6} (d, abbba, 110) \vdash^{Δ6} (d, bbba, 10) \xrightarrow{Δ7} (d, bba, 0) \xrightarrow{} X
							 							    					\end{cases} \\ \\
		    	(1) &\xRightarrow{} \begin{cases}
										    			\vdash^{Δ2} (q, bbba, 11111111110) \text{ (3)} \\
												    	\vdash^{Δ4} (d, bbba, 11111110)  \vdash^{Δ7} (d, bba, 1111110) \vdash^{Δ7} (d, ba, 111110) \vdash^{Δ7} (d, a, 11110) \vdash^{Δ6} (d, e, 1110) \xrightarrow{} X
												     \end{cases}\\\\
     			(3) &\xRightarrow{} \begin{cases}
									     				\vdash^{Δ3} (q, bba, 1111111111110) \text{ (6)} \\
									     				\vdash^{Δ4} (d, bba, 1111111110)  \vdash^{Δ7} (d, ba, 111111110) \vdash^{Δ7} (d, a, 11111110) \vdash^{Δ4} (d, e, 1111110) \xrightarrow{} X
									     			\end{cases}\\ \\	
     			(6) &\xRightarrow{} \begin{cases}
									     				\vdash^{Δ3} (q, ba, 111111111111110) \begin{cases}
									     					\vdash^{Δ3} (q, a, 11111111111111110) \begin{cases}
									     						\vdash^{Δ2} 											(q, e, 1111111111111111110) \xrightarrow{} X\\
									     						\vdash^{Δ4} 										(d, e, 1111111111111110)  		\xrightarrow{} X
									     					\end{cases}\\
									     					\vdash^{Δ4} (d, a, 11111111111110)  \vdash^{Δ6} (d, e, 1111111111110) \xrightarrow{} X
									     				\end{cases} \\
									     				\vdash^{Δ4} (d, ba, 111111111110)  \vdash^{Δ7} (d, a, 11111111110) \vdash^{Δ6} (d, e, 11111111110) \xrightarrow{} X
									     			\end{cases}\\
     			\text{Εφό} &\text{σον η σχέση (6) καταλήγει σε άτοπο, η σχέση (3) καταλήγει και αυτή σε άτοπο και ως άρα και η σχέση (1)}\\\\
     			(2) &\vdash^{Δ7} (q, bbba, 11110) \vdash^{Δ7} (q, bba, 1110) \vdash^{Δ7} (q, ba, 110) \vdash^{Δ7} (q, a, 10) \vdash^{Δ6} (q, e, 0) \vdash^{Δ8} (f, e, 0)   
\end{align*}


\clearpage
\subsection{$L_2 = \{w \in {a, b}^* $: η w περιέχει ακριβώς 2 περισσότερα a από το τριπλάσιο πλήθος β\}}
\noindent
Για την κατασκευή του αυτόματου στοίβας, μπορεί να χρησιμοποιηθεί η μεθοδολογία κατασκευής αυτόματου στοίβας εφόσον είναι γνωστή η γραμματική.

\noindent\\
Η γραμματική της γλώσσας είναι η ίδια γραμματική με αυτή που αναπτύχθηκε στην υποενότητα \ref{sub12}, αλλάζοντας τα σύμβολα a,b μεταξύ τους. Στην συνέχεια ακολουθώντας την αντίστοιχη μεθοδολογία όπως αυτή εμφανίζεται στην θεωρία, υπολογίζεται το αυτόματο στοίβας:\\\\
\begin{minipage}{0.49\textwidth}
	\noindent\\
	\centering
	Γραμματική αναπαράσταση της γλώσσας 
	\begin{align*}
		G = (V, Σ, &R, S)\\
			   V  = \{&a, b, S, S_1\}\\
				Σ = \{&a, b\}\\
				R = \{&S \xrightarrow{} S_1 a S_1 a S_1\\
						 &S_1 \xrightarrow{} bS_1aaa S_1\\
					 	 &S_1 \xrightarrow{} abS_1aa S_1\\
						 &S_1 \xrightarrow{} aabS_1a S_1\\
						 &S_1 \xrightarrow{} aaab S_1\\
						 &S_1 \xrightarrow{} e \}
	\end{align*}
	\vspace{2.4cm}
\end{minipage}
\begin{minipage}{0.49\textwidth}
	\noindent\\
	\centering
	Αυτόματο στοίβας
	\begin{align*}
		Μ = (Κ, Σ, &Γ, Δ, ρ, F)\\
			   K  = \{&p, q\}\\
			   Σ = \{&a, b\}\\
		       Γ = \{&S, a, b\}\\
		       F = \{&q\}\\
		       Δ = \{& Δ_1 : \big(  (p, e, e), (q, S) \big)\\
		      			& Δ_2 : \big(  (q, e, S), (q, S_1aS_1aS_1) \big)\\
	      				& Δ_3 : \big(  (q, e, S_1), (q, bS_1aaaS_1) \big)\\
	      				& Δ_4 : \big(  (q, e, S_1), (q, abS_1aaS_1) \big)\\
	      				& Δ_5 : \big(  (q, e, S_1), (q, aabS_1aS_1) \big)\\
	      				& Δ_6 : \big(  (q, e, S_1), (q, aaabS_1) \big)\\
	      				& Δ_7 : \big(  (q, e, S_1), (q, e) \big)\\
	      				& Δ_8 : \big(  (q, a, a), (q, e) \big)\\
	      				& Δ_9 : \big(  (q, b, b), (q, e) \big)\}
	\end{align*}
\end{minipage}

\noindent\\\\
Υπολογισμός αποδοχής συμβολοσειράς baaaaaaaab σύμφωνα με το παραπάνω αυτόματο:
\begin{align*}
(p, baa&aaaaaab, e)  \vdash^{Δ_1} (q, baaaaaaaab, S)  \vdash^{Δ_2} (q, baaaaaaaab, \textcolor{red}{S_1}aS_1aS_1) \xRightarrow{}\\
&\xRightarrow{}  \begin{cases}
									\vdash^{Δ_3} (q, baaaaaaaab, \textcolor{red}{bS_1aaaS_1}aS_1aS_1) \vdash^{Δ_9}(q, aaaaaaaab, \textcolor{red}{S_1}aaaS_1aS_1aS_1) \xRightarrow{}\\ 
									\vdash^{Δ_4} (q, baaaaaaaab, \textcolor{red}{abS_1aaS_1}aS_1aS_1) \xrightarrow{} X\\
									\vdash^{Δ_5} (q, baaaaaaaab, \textcolor{red}{aabS_1aS_1}aS_1aS_1) \xrightarrow{} X\\
									\vdash^{Δ_6} (q, baaaaaaaab, \textcolor{red}{aaabS_1}aS_1aS_1) \xrightarrow{} X\\
									\vdash^{Δ_7} (q, baaaaaaaab, aS_1aS_1) \xrightarrow{} X
								\end{cases}\\\\
&\xRightarrow{}  \begin{cases}
									\vdash^{Δ_3} (q, aaaaaaaab, \textcolor{red}{bS_1aaaS_1}aaaS_1aS_1aS_1) \xrightarrow{} X\\ 
									\vdash^{Δ_4} (q, aaaaaaaab, \textcolor{red}{abS_1aaS_1}aaaS_1aS_1aS_1) \vdash^{Δ_8} (q, aaaaaaaab, bS_1aaS_1aaaS_1aS_1aS_1) \xrightarrow{} X\\
									\vdash^{Δ_5} (q, aaaaaaaab, \textcolor{red}{aabS_1aS_1}aaaS_1aS_1aS_1) \xRightarrow{}\text{ Ομοίως καταλήγει σε άτοπο μετά από 2 βήματα}\\
									\vdash^{Δ_6} (q, aaaaaaaab, \textcolor{red}{aaabS_1}aaaS_1aS_1aS_1) \xRightarrow{}\text{ Ομοίως καταλήγει σε άτοπο μετά από 3 βήματα}\\
									\vdash^{Δ_7} (q, aaaaaaaab, aaaS_1aS_1aS_1) \xRightarrow{}
								\end{cases}\\\\
&\vdash^{Δ_8} (q, aaaaaaab, aaS_1aS_1aS_1) \vdash^{Δ_8} (q, aaaaaab, aS_1aS_1aS_1) \vdash^{Δ_8} (q, aaaaab, \textcolor{red}{S_1}aS_1aS_1) \xRightarrow{}
\end{align*}
\clearpage
\begin{align*}
&\xRightarrow{}  \begin{cases}
									\vdash^{Δ_3} (q, aaaaab, \textcolor{red}{bS_1aaaS_1}aS_1aS_1) \xrightarrow{} X\\ 
									\vdash^{Δ_4} (q, aaaaab, \textcolor{red}{abS_1aaS_1}aS_1aS_1) \xrightarrow{Δ_8} (q, aaaab, bS_1aaS_1aS_1aS_1) \xrightarrow{} X\\
									\vdash^{Δ_5} (q, aaaaab, \textcolor{red}{aabS_1aS_1}aS_1aS_1) \xRightarrow{}\text{ Ομοίως καταλήγει σε άτοπο μετά από 2 βήματα}\\
									\vdash^{Δ_6} (q, aaaaab, \textcolor{red}{aaabS_1}aS_1aS_1) \xRightarrow{}\text{ Ομοίως καταλήγει σε άτοπο μετά από 3 βήματα}\\
									\vdash^{Δ_7} (q, aaaaab, aS_1aS_1) \xrightarrow{Δ_8} (q, aaaab, \textcolor{red}{S_1}aS_1) \xRightarrow{}
								\end{cases}\\\\
&\xRightarrow{}  \begin{cases}
									\vdash^{Δ_3} (q, aaaab, \textcolor{red}{bS_1aaaS_1}aS_1) \xrightarrow{} X\\ 
									\vdash^{Δ_4} (q, aaaab, \textcolor{red}{abS_1aaS_1}aS_1) \vdash^{Δ_8} (q, aaab, bS_1aaS_1aS_1) \xRightarrow{}\\
									\vdash^{Δ_5} (q, aaaab, \textcolor{red}{aabS_1aS_1}aS_1) \xRightarrow{}\text{ Ομοίως καταλήγει σε άτοπο μετά από 2 βήματα}\\
									\vdash^{Δ_6} (q, aaaab, \textcolor{red}{aaabS_1}aS_1) \xRightarrow{}\text{ Ομοίως καταλήγει σε άτοπο μετά από 3 βήματα}\\
									\vdash^{Δ_7} (q, aaaab, aS_1) \xrightarrow{Δ_8} (q, aaab, \textcolor{red}{S_1}) \xRightarrow{}
								\end{cases}\\\\
&\xRightarrow{}  \begin{cases}
									\vdash^{Δ_3} (q, aaab, \textcolor{red}{bS_1aaaS_1}) \xrightarrow{} X\\ 
									\vdash^{Δ_4} (q, aaab, \textcolor{red}{abS_1aaS_1}) \vdash^{Δ_8} (q, aab, bS_1aaS_1) \xrightarrow{} X\\
									\vdash^{Δ_5} (q, aaab, \textcolor{red}{aabS_1aS_1}) \xRightarrow{}\text{ Ομοίως καταλήγει σε άτοπο μετά από 2 βήματα}\\
									\vdash^{Δ_6} (q, aaab, \textcolor{red}{aaabS_1}) \vdash^{Δ_8} (q, aab, aabS_1) \vdash^{Δ_8} (q, ab, abS_1) \vdash^{Δ_8} (q, b, bS_1) \vdash^{Δ_7} (q, e, \textcolor{red}{S_1})  \xRightarrow{}\\
									\vdash^{Δ_7} (q, aaab, e) \xrightarrow{} X
								\end{cases}\\\\
&\xRightarrow{}  \begin{cases}
									\vdash^{Δ_3} (q, e, \textcolor{red}{bS_1aaaS_1}) \xrightarrow{} X\\ 
									\vdash^{Δ_4} (q, e, \textcolor{red}{abS_1aaS_1}) \xrightarrow{} X\\
									\vdash^{Δ_5} (q, aeaab, \textcolor{red}{aabS_1aS_1}) \xrightarrow{} X\\
									\vdash^{Δ_6} (q, e, \textcolor{red}{aaabS_1}) \xrightarrow{} X\\
									\vdash^{Δ_7} (q, e, e) \xrightarrow{} \text{ Αποδεκτή κατάσταση}
								\end{cases}\\\\
\end{align*}
