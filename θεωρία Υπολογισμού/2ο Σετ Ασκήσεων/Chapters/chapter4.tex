\section{Αναγνώριση Γλωσσών Χωρίς Συμφραζόμενα}
\subsection{Μετατροπή γραμματικής G σε κανονική μορφή Chomsky}
\noindent
Στην δεδομένη γραμματική οι κανόνες είναι οι εξής:
\begin{align*}
	R = \{&S \xrightarrow{} Α\\
			  &Α \xrightarrow{} Μ\\	
			  &Α \xrightarrow{} bbAaT\\
			  &Α \xrightarrow{} ΜaT\\
			  &M \xrightarrow{} a\\
			  &M \xrightarrow{} e\\
			  &T \xrightarrow{} bT\\
			  &T \xrightarrow{} b\}
\end{align*}

\noindent
\textbf{Πρώτο βήμα)} Αλλαγή κανόνων οι οποίοι στο δεξί τους μέρος έχουν περισσότερα από 2 σύμβολα.

\noindent\\
Για κάθε κανόνα που έχει πάνω από 2 σύμβολα στο δεξί του μέρος, δημιουργούνται νέοι κανόνες οι οποίοι στο δεξί τους μέρος έχουν 2 ακριβώς σύμβολα. Οι κανόνες προς αλλαγή είναι οι εξής:\\\\
\begin{minipage}{0.49\textwidth}
	\begin{equation*}
		Α \xrightarrow{} bbAaT \xRightarrow{}  \begin{cases*}
																					A \xrightarrow{} bS_{11}\\
																					S_{11} \xrightarrow{} bS_{12}\\
																					S_{12} \xrightarrow{} AS_{13}\\
																					S_{13} \xrightarrow{} aT
																				\end{cases*} 
	\end{equation*}
\end{minipage}
\begin{minipage}{0.49\textwidth}
	\begin{equation*}
		Α \xrightarrow{} MaT \xRightarrow{}  \begin{cases*}
																				A \xrightarrow{} MS_{21}\\
																				S_{21} \xrightarrow{} aT
																			\end{cases*} 
	\end{equation*}
\end{minipage}

\noindent\\\\
Παρατηρείται πως οι κανόνες $S_{13} \xrightarrow{} aT$ και $S_{21} \xrightarrow{} aT$ οδηγούν στα ίδια σύμβολα, οπότε μπορούν να αντικατασταθούν με έναν, οπότε η νέα γραμματική $G' = (V', Σ, R', S)$ είναι η εξής:
\begin{align*}
	V' = \{&S, A, S_{11}, S_{12}, S_{13}, M, T, a, b\}\\
	R' = \{&S \xrightarrow{} Α\\
			  &Α \xrightarrow{} Μ\\	
			  &\textcolor{red}{Α \xrightarrow{} bS_{11}}\\
			  &\textcolor{red}{S_{11} \xrightarrow{} bS_{12	}}\\
			  &\textcolor{red}{S_{12} \xrightarrow{} AS_{13}}\\
			  &\textcolor{orange}{Α \xrightarrow{} MS_{13}}\\
			  &\textcolor{blue}{S_{13} \xrightarrow{} aT}\\
			  &M \xrightarrow{} a\\
			  &M \xrightarrow{} e\\
			  &T \xrightarrow{} bT\\
			  &T \xrightarrow{} b\}
\end{align*}


\noindent
\textbf{Δεύτερο Βήμα)} Απαλοιφή κενών κανόνων

\noindent\\
Υπολογίζεται το διάνυσμα $\mathcal{E}$ στο οποίο τοποθετούνται όλα τα σύμβολα για τα οποία υπάρχει κενός κανόνας (κανόνας που οδηγεί σε e) ή κανόνες που μπορούν να οδηγήσουν σε κενό κανόνα.

\noindent\\
Έτσι το διάνυσμα $\mathcal{E}$ περιέχει τα εξής σύμβολα:
\begin{equation*}
	\mathcal{E} = \{S, A, M\} \text{ εφόσον} \begin{cases*}
																				S \xrightarrow{} A \xrightarrow{} M \xrightarrow{} e\\
																				A \xrightarrow{} M \xrightarrow{} e\\
																				M \xrightarrow{} e
																			\end{cases*}
\end{equation*}
\noindent
Ο κενός κανόνας ($M \xrightarrow{} e$) απαλείφεται ενώ ο ελέγχονται όλοι οι κανόνες που περιέχουν ένα από τα τρία σύμβολα και για κάθε έναν από αυτούς, δημιουργούνται οι αντίστοιχοι νέοι:
\begin{align*}
 S \xrightarrow{} Α &\xRightarrow{} \text{ Μένει όπως έχει}\\
A \xrightarrow{} Μ &\xRightarrow{} \text{ Μένει όπως έχει}\\
S_{12} \xrightarrow{} AS_{13} &\xRightarrow{} \begin{cases*}
																						S_{12} \xrightarrow{} AS_{13}\\
																						S_{12} \xrightarrow{} S_{13}
																					\end{cases*}\\
	Α \xrightarrow{} MS_{13} &\xRightarrow{}  	\begin{cases*}
																				 		S_{12} \xrightarrow{} MS_{13}\\
																				 		S_{12} \xrightarrow{} S_{13}			 	
																				 	\end{cases*}\\
\end{align*}

\noindent\\
Έτσι η νέα γραμματική $G'' = (V'', Σ, R'', S)$ είναι η εξής:
\begin{align*}
	V'' = \{&S, A, S_{11}, S_{12}, S_{13}, M, T, a, b\}\\
	R'' = \{&S \xrightarrow{} Α\\
	&Α \xrightarrow{} Μ\\	
	&Α \xrightarrow{} bS_{11}\\
	&Α \xrightarrow{} MS_{13}\\
	&\textcolor{orange}{A \xrightarrow{} S_{13}}\\
	&S_{11} \xrightarrow{} bS_{12}\\
	&S_{12} \xrightarrow{} AS_{13}\\
	&\textcolor{red}{S_{12} \xrightarrow{} S_{13}}\\
	&S_{13} \xrightarrow{} aT\\
	&M \xrightarrow{} a\\
	&T \xrightarrow{} bT\\
	&T \xrightarrow{} b\}
\end{align*}

\noindent\\
\textbf{Τρίτο βήμα )} Απαλοιφή μικρών κανόνων

\noindent\\
Για την απαλοιφή κανόνων που οδηγούν σε ένα μόνο σύμβολο, απαιτείται ο υπολογισμός των διανυσμάτων $\mathcal{D}$ για κάθε ένα από τα μη τερματικά σύμβολα (Τα τερματικά περιέχουν μόνο το εαυτό τους).

\noindent\\
Στο διάνυσμα $\mathcal{D}$ τοποθετούνται, πέραν του ίδιου του συμβόλου, όλα τα μονά σύμβολα στα οποία μπορεί να καταλήξει σύμφωνα με τους κανόνες γραμματικής. Τα διανύσματα $\mathcal{D}$ υπολογίζονται ως εξής:
\begin{align*}
	&\mathcal{D} (S) = \{S, A, M, a, S_{21}\} \\
	&\mathcal{D} (A) = \{A, M, a, S_{21} \} \\
	&\mathcal{D} (S_{11}) = \{S_{11}\} \\
	&\mathcal{D} (S_{12}) = \{S_{12}, S_{13}\}\\
	&\mathcal{D} (S_{13}) = \{S_{13}\}\\
	&\mathcal{D} (M) = \{M, a\}\\
	&\mathcal{D} (T)  = \{T, b\}
\end{align*}

\noindent\\
Σύμφωνα με  τα παραπάνω διανύσματα, τα σύμβολα S, A, $S_{12}$, M, T έχουν κανόνες οι οποίοι οδηγούν σε ένα μόνο σύμβολο. Έτσι, ομοίως με το προηγούμενο ερώτημα, θα γίνει απαλοιφή αυτών και δημιουργία νέων.

\begin{align*}
	&\text{Κανόνας }A \xrightarrow{} \textcolor{red}{M}S_{13} \text{: προστίθεται ο εξής κανόνας } A \xrightarrow{} aS_{13} \\\\
	&\text{Κανόνας }S_{11} \xrightarrow{} b\textcolor{red}{S_{12}} \text{: προστίθεται ο κανόνας } S_{11}\xrightarrow{} bS_{13}\\
	&\text{Κανόνας }S_{12} \xrightarrow{} \textcolor{red}{A}S_{13} \text{: προστίθενται οι εξής κανόνες}
			\begin{cases*}
				S_{12}\xrightarrow{} MS_{13}\\
				S_{12}\xrightarrow{} aS_{13}\\
				S_{12}\xrightarrow{} S_{13}S_{13}
			\end{cases*}\\
	&\text{Κανόνας }S_{13} \xrightarrow{} a\textcolor{red}{T} \text{: προστίθεται ο εξής κανόνας } S_{13}\xrightarrow{} ab\\\\
	&\text{Κανόνας }T \xrightarrow{} b\textcolor{red}{T} \text{: προστίθεται ο εξής κανόνας } T \xrightarrow{} bb
\end{align*}
\noindent\\
Εφόσον, απαλείφεται ο κανόνας $M \xrightarrow{} a$, όλοι οι κανόνες που έχουν το σύμβολο M στο δεξί τους χέρι απαλείφονται.

\noindent\\
Εφόσον απαλείφεται ο κανόνας $S \xrightarrow{} A$ είναι απαραίτητο να δημιουργηθούν νέοι ώστε το αυτόματο να μπορεί να εκκινήσει. Οι κανόνες αυτοί προκύπτουν μέσω σύγκρισης του διανύσματος $\mathcal{D}(S)$ και των ήδη υπάρχοντων κανόνων. Στην περίπτωση που κάποιο από τα σύμβολα του $\mathcal{D}$ (S) υπάρχει στο αριστερό μέρος κάποιου κανόνα, δημιουργείται νέος κανόνας ο οποίος έχει αριστερά την τιμή $S$.
\begin{equation*}
	S \xrightarrow{} A \xRightarrow{}  	 \begin{cases*}
																		\textcolor{orange}{S \xrightarrow{} bS_{11}} \text{ }(A \xrightarrow{} bS_{11})\\
																		\textcolor{orange}{S \xrightarrow{} aS_{13}} \text{ }(A \xrightarrow{} aS_{13})\\
																		\textcolor{orange}{S \xrightarrow{} aT}  \hspace{0.4cm} (S_{13} \xrightarrow{} aT)\\
																		\textcolor{orange}{S \xrightarrow{} ab} \hspace{0.5cm} (S_{13} \text{ } \xrightarrow{} ab)\\
																	\end{cases*}
\end{equation*}
\noindent\\
Έτσι η νέα γραμματική $G''' = (V''', Σ, R''', S)$ είναι η εξής:
\begin{align*}
	V''' = \{&S, A, S_{11}, S_{12}, S_{13}, T, a, b\}\\
	R''' = \{&\textcolor{orange}{S \xrightarrow{} bS_{11}}\\
				&\textcolor{orange}{S \xrightarrow{} aS_{13}}\\
				&\textcolor{orange}{S \xrightarrow{} aT	}\\
				&\textcolor{orange}{S \xrightarrow{} ab}\\
				&Α \xrightarrow{} bS_{11}\\
				&\textcolor{red}{A \xrightarrow{} aS_{13}}\\
				&S_{11} \xrightarrow{} bS_{12}\\	
				&\textcolor{red}{S_{11} \xrightarrow{} bS_{13}}\\	
				&S_{12} \xrightarrow{} AS_{13}\\
				&\textcolor{red}{S_{12} \xrightarrow{} aS_{13}}\\
				&\textcolor{red}{S_{12} \xrightarrow{} S_{13}S_{13}}\\
				&S_{13} \xrightarrow{} aT\\
				&\textcolor{red}{S_{13} \xrightarrow{} ab}\\
				&T \xrightarrow{} bT \\
				&\textcolor{red}{T \xrightarrow{} bb}\}
\end{align*}

\clearpage
\noindent
Η τελική γραμματική σε μορφή Chomsky είναι η $G''' = (V''', Σ, R''', S)$ είναι η εξής:
\begin{align*}
	V''' = \{&S, A, S_{11}, S_{12}, S_{13}, T, a, b\}\\
	Σ = \{&a, b\}\\
	R''' = \{&S \xrightarrow{} bS_{11} \hspace{1cm} [1] \\
				&S \xrightarrow{} aS_{13} \hspace{1cm} [2] \\
				&S \xrightarrow{} aT \hspace{1.3cm} [3] \\
				&S \xrightarrow{} ab \hspace{1.4cm} [4] \\
				&Α \xrightarrow{} bS_{11} \hspace{1cm} [5] \\
				&A \xrightarrow{} aS_{13} \hspace{1cm} [6] \\
				&S_{11} \xrightarrow{} bS_{12} \hspace{0.8cm} [7] \\	
				&S_{11} \xrightarrow{} bS_{13} \hspace{0.8cm} [8] \\	
				&S_{12} \xrightarrow{} AS_{13} \hspace{0.7cm} [9] \\
				&S_{12} \xrightarrow{} aS_{13} \hspace{0.7cm} [10] \\
				&S_{12} \xrightarrow{} S_{13}S_{13} \hspace{0.3cm}[11] \\
				&S_{13} \xrightarrow{} aT \hspace{0.9cm} [12] \\
				&S_{13} \xrightarrow{} ab \hspace{1cm} [13] \\
				&T \xrightarrow{} bT \hspace{1.2cm} [14] \\
				&T \xrightarrow{} bb \hspace{1.3cm} [15] \}
\end{align*}	

\noindent\\
\subsubsection{πίνακας Συντακτικής Ανάλυσης και Συντακτικό Δένδρο}

\noindent\\
Αρχικά είναι απαραίτητο να ελεγχθεί αν η συμβολοσειρά εισόδου  $ w = bbababb$ παράγεται από την αρχική γραμματική:
\begin{equation*}
	S \vdash A \vdash bbAaT \vdash bbMaTaT \vdash bbaTaT \vdash bbabaT \vdash bbababT \vdash bbababb
\end{equation*}

\noindent\\
Εφόσον η είσοδος παράγεται από την αρχική γραμματική θα πρέπει να παράγεται και από την τελική γραμματική σε μορφή Chomsky. 

\noindent\\
Ο πίνακας της συντακτικής ανάλυσης και τo συντακτικό δένδρο φαίνονται παρακάτω ενώ σε τετράγωνες αγκύλες φαίνεται
ο αριθμός του κανόνα:
	\begin{table}[h]
		\centering
		\begin{tabular}{cccccc|c|}
			\cline{7-7}
			&   &    &    &    &    & b                                                        \\ \cline{6-7} 
			&   &    &    &\multicolumn{1}{c|}{}  & b                                                        & T [15]                                                   \\ \cline{5-7} 
			&                             &                        & \multicolumn{1}{c|}{}                                                         & \multicolumn{1}{c|}{a} & \begin{tabular}[c]{@{}c@{}}$S_{13}$ [13]\\ S [4]\end{tabular} & \begin{tabular}[c]{@{}c@{}}$S_{13}$ [12]\\ S [3]\end{tabular} \\ \cline{4-7} 
			&                             & \multicolumn{1}{c|}{}  & \multicolumn{1}{c|}{b}                                                        & \multicolumn{1}{c|}{0} & $S_{11}$ [8]                                                  & $S_{11}$ [8]                                                  \\ \cline{3-7} 
			& \multicolumn{1}{c|}{}       & \multicolumn{1}{c|}{a} & \multicolumn{1}{c|}{\begin{tabular}[c]{@{}c@{}}$S_{13}$ [13]\\ S [4]\end{tabular}} & \multicolumn{1}{c|}{0} &$S_{13}$ [11]                                                 & $S_{13}$[11]                                                 \\ \cline{2-7} 
			\multicolumn{1}{c|}{}   & \multicolumn{1}{c|}{b}      & \multicolumn{1}{c|}{0} & \multicolumn{1}{c|}{$S_{11}$ [8]}                                                  & \multicolumn{1}{c|}{0} & $S_{11}$ [7]                                                  & $S_{11}$ [8]                                                  \\ \hline
			\multicolumn{1}{|c|}{b} & \multicolumn{1}{c|}{T [15]} & \multicolumn{1}{c|}{0} & \multicolumn{1}{c|}{\begin{tabular}[c]{@{}c@{}}A [5]\\ S [1]\end{tabular}}    & \multicolumn{1}{c|}{0} & \begin{tabular}[c]{@{}c@{}}A [5]\\ S[1]\end{tabular}     & S [1]                                                    \\ \hline
		\end{tabular}
	\end{table}
\noindent\\\\\\
\begin{center}
	\begin{tikzpicture}[  level 1/.style={level distance=15mm,sibling distance=30mm},
										level 2/.style={level distance=15mm,sibling distance=30mm},
										level 3/.style={level distance=15mm, sibling distance=30mm},
										level 4/.style={level distance=15mm, sibling distance=25mm},
										font=\scriptsize,inner sep=1pt,every node/.style={minimum size=3ex}, text=black, >=latex ]
\node (1){\(S\)} 
child { node {b}}
child { node {$S_{11} [8]$}
	child { node {b}}
	child { node {$S_{13} [11]$}
		child { node {$S_{13} [13]$}
			child { node {$a$}}
			child { node {$b$}}
		}
		child { node {$S_{13} [12]$}
			child { node {$a$}}
			child { node {$T [15]$}
				child { node {$b$}}
				child { node {$b$}}
			}
		}
	}
};
\end{tikzpicture}
\end{center}