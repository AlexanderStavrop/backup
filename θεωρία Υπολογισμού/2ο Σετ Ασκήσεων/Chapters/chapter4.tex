\section{Αναγνώριση Γλωσσών Χωρίς Συμφραζόμενα}
\subsection{Μετατροπή γραμματικής G σε κανονική μορφή Chomsky}
\noindent
Στην δεδομένη γραμματική οι "ΠΩΣ ΛΈΓΟΝΤΑΙ ΤΑ R" είναι οι εξής:
\begin{align*}
	R = \{&S \xrightarrow{} Α\\
			  &Α \xrightarrow{} Μ\\	
			  &Α \xrightarrow{} bbAaT\\
			  &Α \xrightarrow{} ΜaT\\
			  &M \xrightarrow{} a\\
			  &M \xrightarrow{} e\\
			  &T \xrightarrow{} bT\\
			  &T \xrightarrow{} b\}
\end{align*}

\noindent\\
\textbf{Πρώτο βήμα)} Αλλάζουμε οτι είναι μεγαλύτερο του 2 στα δεξια φαδφασδφασδξηφαλκσδξφηακσδξφλ!!!!!!!!!!!!!!!!!\\\\

\begin{minipage}{0.49\textwidth}
	\begin{equation*}
		Α \xrightarrow{} bbAaT \xRightarrow{}  \begin{cases*}
																					A \xrightarrow{} bS_{11}\\
																					S_{11} \xrightarrow{} bS_{12}\\
																					S_{12} \xrightarrow{} AS_{13}\\
																					S_{13} \xrightarrow{} aS_{14}\\
																					S_{14} \xrightarrow{} T\\
																				\end{cases*} 
	\end{equation*}
\end{minipage}
\begin{minipage}{0.49\textwidth}
	\begin{equation*}
		Α \xrightarrow{} MaT \xRightarrow{}  \begin{cases*}
																				A \xrightarrow{} MS_{21}\\
																				S_{21} \xrightarrow{} aS_{22}\\
																				S_{22} \xrightarrow{} T\\
																			\end{cases*} 
	\end{equation*}
\end{minipage}
