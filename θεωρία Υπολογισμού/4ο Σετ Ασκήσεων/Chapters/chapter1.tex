\section{Οι κλάσεις P και NP}

\subsection{Αποδείξτε ότι αν $L_1 \cap L_2 \in \mathcal{N}P$ και $\overline{L_1} \cup \overline{L_2} \in \mathcal{N}P$, τότε $\overline{L_1 \cap L_2} \in P$}
\subsection{Δείξτε ότι  το πρόβλημα 2023-MaxSAT ανήκει στην κλάση $\mathcal{N}P$}

Το πρόβλημα MAXSAT είναι πρόβλημα τύπου $\mathcal{N}P$, όπως είναι γνωστό από την θεωρία. Εφόσον, γίνεται αναφορά στο πρόβλημα 2023-MaxSAT, μπορεί να αποδεχθεί πως και αυτό ανήκει στην κλάση$\mathcal{N}P$ μέσω εφαρμογής της πολυωνυμικής  αναγωγής.

\noindent\\
Η πολυωνυμική αναγωγή εφαρμόζεται μέσω μίας ντετερμινιστικής μηχανής Turing Μ η οποία δέχεται στη είσοδό μία λογική πρόταση CNF υπό μορφή συμβολοσειράς η οποία περιέχει ακριβώς 2023 ανά συνθήκη.  Η μηχανή αλλάζει τον αριθμό των στοιχείων ανά συνθήκη ώστε να παράξει την αντίστοιχη ισοδύναμη λογική πρόταση και η απόφαση για το πρόβλημα MAXSAT προκύπτει μέσω μηχανής απόφαση. Έτσι, η απόφαση για τον 2023-MAXSAT πρόβλημα προκύπτει  σε μη-ντετερμινιστικά πολυωνυμικό χρόνο, δηλαδή το πρόβλημα 2023-MaxSAT ανήκει και αυτό στην κλάση $\mathcal{N}P$





\begin{figure}[h]
	\centering
	\begin{subfigure}{.5\textwidth}
		\centering
		\includegraphics[width=0.95\textwidth]{Images/university.png}
		\caption{subcaption 1}
		\label{circ:1_phase_D}
	\end{subfigure}%
	\begin{subfigure}{.5\textwidth}
		\centering
		\includegraphics[width=\textwidth]{Images/university.png}
		\caption{subcaption 1}
		\label{circ:1_phase_Th}
	\end{subfigure}
	\caption{Caption 1}
\end{figure}

\begin{equation}
	a \leq \tan^{-1} \left(\frac{\omega L}{R} \right) \label{eq:Q11_V_o}
\end{equation}
