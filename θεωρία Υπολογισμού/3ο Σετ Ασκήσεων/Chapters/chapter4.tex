\noindent
\section{Μη επιλυσιμότητα}

\subsection{Δύο μηχανές Turing M1, M2, υπάρχει έστω και μία συμβολοσειρά για την οποία τερματίζουν και οι δύο μηχανές;}

Για την απάντηση της ερώτησης, μπορεί να χρησιμοποιηθεί το εξής γνωστό πρόβλημα:
\begin{equation*}
	L = \{"M'" : M' \text{ τερματίζει για κάποια είσοδο}\}
\end{equation*}

το οποίο πρόβλημα μπορεί πολύ εύκολα να κωδικοποιηθεί αξιοποιώντας δύο μηχανές Turing:
\begin{equation*}
	L_1 = \{"M_1" "M_2" : \text{$M_1, Μ_2$ τερματίζουν για την ίδια ακριβώς συμβολοσειρά εισόδου} \}
\end{equation*}


\noindent\\
Υποθέτοντας πως το πρόβλημα είναι επιλύσιμο, θα πρέπει η γλώσσα $L_1$ να είναι αναδρομική, οπότε υπάρχει μηχανή Turing $Μ_a$ η οποία αποφασίζει γι αυτή την γλώσσα μέσω της εξής αναγωγής:
\begin{equation*}
	"Μ'" \xrightarrow{} "Μ'""M_2" \text{ όπου η $M_2$ τερματίζει για κάθε είσοδο}
\end{equation*}

\noindent\\
Η διαδικασία εκτέλεσης ξεκινά με την μηχανή M, να υπολογίζει την αναδρομική σχέση $Μ'""M_2"$, δίνει το αποτέλεσμα ως είσοδο στην μηχανή $M_a$ η οποία με την σειρά της δέχεται την συμβολοσειρά $"Μ'""M_2"$ δεδομένου του ότι οι εκάστοτε μηχανές τερματίζουν. Την τελική απόφαση για την γλώσσα την παίρνει η  μηχανή Μ κάτι το οποίο όμως δεν ισχύει άρα, καταλήγουμε σε άτοπο, οπότε το πρόβλημα είναι μη επιλύσιμο.\\



\subsection{Δύο μηχανές Turing M1, M2, τερματίζει η μηχανή M2 για αυστηρά λιγότερες εισόδους από την M1}
Για την απάντηση της ερώτησης, μπορεί να χρησιμοποιηθεί το εξής γνωστό πρόβλημα:
\begin{equation*}
	L = \{"M" : M \text{ τερματίζει για κάποια είσοδο}\}
\end{equation*}

\noindent\\
Αντίστοιχα με την  M, όλες οι "Μ" αναπαραστάσεις ανήκουν στην γλώσσα L και ημιαποφασίζουν για κάποια μη κενή γλώσσα, οπότε το πρόβλημα μπορεί πολύ εύκολα να κωδικοποιηθεί αξιοποιώντας δύο μηχανές Turing:
\begin{equation*}
	L_1 = \{"M_1" "M_2" : \text{$M_1$ τερματίζει για περισσότερες εισόδους σε σχέση με την $Μ_2$} \}
\end{equation*}


\noindent\\
Υποθέτοντας πως το πρόβλημα είναι επιλύσιμο, θα πρέπει η γλώσσα $L_1$ να είναι αναδρομική, οπότε υπάρχει μηχανή Turing $Μ_a$ η οποία αποφασίζει γι αυτή την γλώσσα μέσω της εξής αναγωγής:
\begin{equation*}
	"Μ" \xrightarrow{} "Μ""M_3" \text{ όπου η $M_3$ δεν τερματίζει (infinte loop)}
\end{equation*}

\noindent
Η παραπάνω αναπαράσταση είναι αναδρομική καθώς η $M_3$ χρησιμοποιείται εύκολα σε συνέχεια της M. Αντίστοιχα με την προηγούμενη περίπτωση, η $Μ_3$ δέχεται ως είσοδο την M η οποία με την σειρά της υπολογίζει την παραπάνω αναδρομική σχέση. Στην συνέχεια, αξιοποιώντας την μηχανή $M_a$ με είσοδο $"M"M_3"$ την οποία είσοδο αποδέχεται στην περίπτωση όπου η Μ τερματίζει για περισσότερες εισόδους από την $M_3$. Και πάλι, την τελική απόφαση για την γλώσσα την παίρνει η  μηχανή $Μ_3$ κάτι το οποίο όμως δεν ισχύει άρα, καταλήγουμε σε άτοπο, οπότε το πρόβλημα είναι μη επιλύσιμο.


