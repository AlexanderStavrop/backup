\section{Κανονικές γλώσσες}
\subsection[Ερώτημα α]{\textbf{α) }} Η τομή μιας κανονικής γλώσσας με μια μη κανονική γλώσσα είναι πάντα κανονική γλώσσα.

\noindent\\
\textbf{Λάθος} Ακολουθεί αντιπαράδειγμα:

\noindent\\
Έστω μία κανονική γλώσσα $L_1 = \{a^*\}$ και μία μη κανονική γλώσσα $L_2 = \{a^n b^n, n > 0\}$ και έστω ότι η τομή των δύο γλωσσών παράγει μία κανονική γλώσσα:
\begin{equation*}
	L_1 \cap L_2 = \{a^n, n>0\}
\end{equation*}
Άτοπο το αποτέλεσμα της τομής δεν είναι κανονικό.\\

\subsection[Ερώτημα β]{\textbf{β) }} Το συμπλήρωμα μιας πεπερασμένης γλώσσας είναι πάντα κανονική γλώσσα.

\noindent\\
\textbf{Σωστό}

\noindent\\
Κάθε πεπερασμένη γλώσσα είναι κανονική και επίσης το συμπλήρωμα μίας κανονικής γλώσσας είναι επίσης κανονική γλώσσα λόγω κλειστότητας της πράξης.\\

\subsection[Ερώτημα γ]{\textbf{γ) }} Κάθε μη κανονική γλώσσα περιέχει μη μετρήσιμο πλήθος συμβολοσειρών.

\noindent\\
\textbf{Λάθος}

\noindent\\
Κάθε κανονική γλώσσα είναι υποσύνολο του $\Sigma^*$ το οποίο είναι ένα μετρήσιμα άπειρο σύνολο. 

