\section{Κανονικές Εκφράσεις}

\subsection[Ερώτημα α]{\textbf{α) }} L = \{w $\in \{a, b\}^*$ : η w περιέχει ακριβώς 2 εμφανίσεις του a και άρτιο αριθμό από b\}
\begin{equation*}
	L = \mathcal{L} \Bigg( (bb)^* \bigg( \Big( (ab \cup ba) (bb)^*(ab \cup ba) \Big)  \bigcup    \Big(aa\Big)  \bigg) (bb)^*\Bigg)
\end{equation*}

\subsection[Ερώτημα β]{\textbf{β) }} L = \{w $\in {a, b}^*$ : η w αρχίζει και τελειώνει με το ίδιο σύμβολο και έχει περιττό μήκος\}
\begin{equation*}
	L = \mathcal{L} \Bigg( \bigg(  a \Big( (a \cup b)(a \cup b) \Big)^* (a \cup b) a \bigg)  \bigcup   \bigg( b \Big( (a \cup b)(a \cup b) \Big)^* (a \cup b) b \bigg) \Bigg)
\end{equation*}

\subsection[Ερώτημα γ]{\textbf{γ) }} L = \{w $\in {a, b}^*$ : το πλήθος των a στην w είναι 4k + 1 (k $\geq$ 0) και δεν εμφανίζονται συνεχόμενα a\}
\begin{equation*}
	L = \mathcal{L} \bigg( b^*(a b^+a b^+a b^+a b^+)^*ab^* \bigg)
\end{equation*}

\noindent\\
\section{Πεπερασμένα αυτόματα}

\subsection[Ερώτημα α]{\textbf{α) }} L = \{w $\in \{a, b\}^*$ : η w έχει περιττό μήκος και τελειώνει σε ab\}
\newcommand\leftX{-1.5}
\newcommand\rightX{1.5}
\newcommand\topY{2}
\newcommand\midY{0}
\newcommand\bottomY{-2}

\noindent\\\\
\begin{center}
	\begin{tikzpicture}[>=stealth',shorten >=1pt,auto,node distance=2.7 cm, scale = 1, transform shape]
			\node[initial,state, thick] 		  (q0)   							   {$q_0$};
			\node[state, thick]           			 (q2) [right of=q0]       	{$q_2$};
			\node[state, thick]       				 (q1) [above of=q2]       {$q_1$};
			\node[state,accepting, thick]	(q3) [below of=q2]       {$q_3$};
			
		\path[->] (q0) edge [thick, bend right=-10]   node [align=center]  {$a,b$} (q1)
						(q1) edge [thick, bend right=-10]    node [align=center]  {$b$} (q0)
						(q1) edge [thick, bend right=-10]    node [align=center]  {$a$} (q2)
						(q2) edge [thick, bend right=-10]    node [align=center]  {$a$} (q1)
						(q2) edge [thick]    							  node [align=center]  {$b$} (q3)
						(q3) edge [thick]    							  node [align=center]  {$a,b$} (q0);
	\end{tikzpicture}
\end{center}

\subsection[Ερώτημα β]{\textbf{β) }} L = \{w $\in \{a, b\}^*$ : η w έχει άρτιο μήκος και το πλήθος των b είναι 3m + 1 για κάποιο m $\geq$ 0\}

\begin{center}
	\begin{tikzpicture}[>=stealth',shorten >=1pt,auto,node distance=2.5 cm, scale = 1, transform shape]
		\node[initial,state, thick]  			 (q0)   						  {$q_0$};
		\node[state, thick, accepting]     (q1) [right of=q0]     {$q_1$};
		\node[state, thick]           				(q2) [right of=q1]     {$q_2$};
		\node[state, thick]           				(q3) [right of=q2]     {$q_3$};
		\node[state, thick]							(q4) [below of=q0]  {$q_4$};
		\node[state, thick]							(q5) [below of=q1]  {$q_5$};
		\node[state, thick]							(q6) [below of=q2]  {$q_6$};
		\node[state, thick]							(q7) [below of=q3] 	{$q_7$};
		
		\path[->] (q0) edge [thick, bend right=-10]     node [align=center]  {$a$} (q4)
						 (q4) edge [thick, bend right=-10]     node [align=center]  {$a$} (q0)
						 (q0) edge [thick]    							    node [at start, yshift=-13pt, xshift=10pt]  {$b$} (q5)
						 (q4) edge [thick]    							    node [at start, yshift=-3pt, xshift=20pt]  {$b$} (q1)
						 (q1) edge [thick, bend right=-10]     node [align=center]  {$a$} (q5)
						 (q5) edge [thick, bend right=-10]     node [align=center]  {$a$} (q1)
						 (q1) edge [thick]    							    node [at start, yshift=-13pt, xshift=10pt]  {$b$} (q6)
						 (q5) edge [thick]    							    node [at start, yshift=-3pt, xshift=20pt]  {$b$} (q2)
						 (q2) edge [thick, bend right=-10]     node [align=center]  {$a$} (q6)
						 (q6) edge [thick, bend right=-10]     node [align=center]  {$a$} (q2)
						 (q2) edge [thick]    							    node [align=center]  {$b$} (q3)
						 (q6) edge [thick]    							    node [align=center]  {$b$} (q7)
						 (q3) edge [thick, bend right=40]     node [above, align=center]  {$a$} (q1)
						 (q7) edge [thick, bend right=-40]     node [align=center]  {$a$} (q5)
 						 (q3) edge [thick, bend right=-10]     node [align=center]  {$a$} (q7)
						 (q7) edge [thick, bend right=-10]     node [align=center]  {$a$} (q3);
	\end{tikzpicture}
\end{center}

\clearpage
\subsection[Ερώτημα γ]{\textbf{γ) }} L = \{w $\in \{a, b\}^*$ : το πρώτο σύμβολο της w έχει περιττό αριθμό εμφανίσεων\}

\begin{figure}[h]
	\centering
	\begin{tikzpicture}[>=stealth',shorten >=1pt,auto, scale = 1, transform shape]
		\node[initial,state, thick]  			 (q0)   						  			  							 {$q_0$};
		\node[state, thick, accepting]     (q1) [above right=0.3cm and 1cm of q0]    {$q_1$};
		\node[state, thick]           				(q2) [right=2cm  of q1]     			   				  {$q_2$};
		\node[state, thick, accepting]     (q3) [below right=0.3cm and 1cm of  q0]   {$q_3$};
		\node[state, thick]							(q4) [right= 2cm of q3]  			   			 		  {$q_4$};
		
		\path[->] (q0) edge [thick]     							 node [align=center]  {$a$} (q1)
						 (q0) edge [thick, below]     				  node [align=center, xshift=-7pt, yshift=-1pt]  {$b$} (q3)
						 
						 (q1) edge [thick, bend right=-10 ]    node [align=center]  {$a$} (q2)
 						 (q2) edge [thick, bend right=-10 ]    node [align=center]  {$a$} (q1)
 						 (q1) edge [thick,loop above]   		  node [align=center]  {$b$} (q1)
 						 (q2) edge [thick,loop above]   		  node [align=center]  {$b$} (q2)
 						 
 						 (q3) edge [thick, bend right=-10 ]    node [align=center]  {$b$} (q4)
 						 (q4) edge [thick, bend right=-10 ]    node [align=center]  {$b$} (q3)
 						 (q3) edge [thick,loop above]   		  node [align=center]  {$a$} (q3)
 						 (q4) edge [thick,loop above]   		  node [align=center]  {$a$} (q4);
	\end{tikzpicture}
\end{figure}

\section{Μη ντετερμινισμός και κανονικότητα αυτομάτων}

\begin{figure}[h]
	\centering
	\begin{tikzpicture}[>=stealth',shorten >=1pt,auto, node distance=2.3 cm, scale = 1, transform shape]
		\node[initial,state, thick]  			 (q1)   						{$q_1$};
		\node[state, thick, accepting]     (q2) [right of=q1]   {$q_2$};
		\node[state, thick]           				(q3) [right of=q2]   {$q_3$};
		\node[state, thick]           				(q4) [right of=q3]   {$q_4$};
		
		\path[->] (q1) edge [thick]     							 node [align=center]  {$b$} (q2)
						 (q1) edge [thick, bend right=-40 ]    node [align=center]  {$e$} (q3)
						(q2) edge [thick, bend right=-10 ]    node [align=center]  {$a$} (q3)
						(q3) edge [thick, bend right=-10 ]    node [align=center]  {$b$} (q2)
						(q3) edge [thick]     							   node [align=center]  {$a$} (q4)
						(q3) edge [thick,loop above]   		     node [align=center]  {$a$} (q3)
						(q4) edge [thick, bend right=-25]     node [align=center]  {$e$} (q1);
	\end{tikzpicture}
	\caption{Μη ντετερμινιστικό αυτόματο Μ}
\end{figure}

\subsection[Ερώτημα α]{\textbf{α) }}Κατασκευάστε αναλυτικά ένα ισοδύναμο ντετερμινιστικό αυτόματο $Μ^,$ ???????????
\begin{figure}[h]
	\centering
	\begin{tikzpicture}[>=stealth',shorten >=1pt,auto, scale = 1, transform shape]
		\node[initial,state, thick, shape=ellipse, accepting]	 (q0)	 																	  {$\{q_1, q_2\}$};
		\node[state, thick, shape=ellipse]									 (q1)    [above right=0.5cm and 1cm of q0]     {$\{q_1, q_3, q_4\}$};
		\node[state, thick, accepting]           							    (q3) 	[right=2cm  of q1]     			   				  	  {$\{q_3\}$};
		\node[state, thick]     				   									 	(q2)    [below right=0.2cm and 1.6cm of  q0]  {$\{q_2\}$};
		\node[state, thick]										    					(q4) 	[right= 3cm of q2]  			   			 		 {$\{ \}$};
		
		\path[->] (q0) edge [thick]     				   		 	node [align=center]  {$a$} (q1)
						(q0) edge [thick, below]     				 node [align=center, xshift=-7pt, yshift=-1pt]  {$b$} (q2)
						(q1) edge [thick,loop above]    		 node [align=center]  {$a$} (q1)
						(q1) edge [thick]    			   		  		   node [align=center]  {$b$} (q2)
						(q3) edge [thick, above]    				 node [align=center]  {$a$} (q1)
						(q3) edge [thick, bend right=-10 ]    node [align=center]  {$b$} (q2)
						(q2) edge [thick, bend right=-10 ]    node [align=center]  {$a$} (q3)
						(q2) edge [thick, below]    				 node [align=center]  {$b$} (q4)
						(q4) edge [thick, loop above]   		 node [align=center]  {$a,b$} (q4);
	\end{tikzpicture}
\end{figure}

\clearpage
%%%%%%%%%%%%%%%%%%%%%%%%%%%%%%% Vanila state %%%%%%%%%%%%%%%%%%%%%%%%%%%%%%%
\begin{center}
	\begin{tikzpicture}[>=stealth',shorten >=1pt,auto, node distance=2.3 cm, scale = 1, transform shape]
		\node[initial,state, thick]  			 (q1)   						{$q_1$};
		\node[state, thick, accepting]     (q2) [right of=q1]   {$q_2$};
		\node[state, thick]           				(q3) [right of=q2]   {$q_3$};
		\node[state, thick]           				(q4) [right of=q3]   {$q_4$};
		
		\path[->] (q1) edge [thick]     							 node [align=center]  {$b$} (q2)
		(q1) edge [thick, bend right=-40 ]    node [align=center]  {$e$} (q3)
		(q2) edge [thick, bend right=-10 ]    node [align=center]  {$a$} (q3)
		(q3) edge [thick, bend right=-10 ]    node [align=center]  {$b$} (q2)
		(q3) edge [thick]     							   node [align=center]  {$a$} (q4)
		(q3) edge [thick,loop above]   		     node [align=center]  {$a$} (q3)
		(q4) edge [thick, bend right=-25]     node [align=center]  {$e$} (q1);
	\end{tikzpicture}
\end{center}

%%%%%%%%%%%%%%%%%%%%%%%%%%%%%%%% Init state %%%%%%%%%%%%%%%%%%%%%%%%%%%%%%%%
\begin{center}
	\begin{tikzpicture}[>=stealth',shorten >=1pt,auto, node distance=2.3 cm, scale = 1, transform shape]
		\node[initial,state, thick]  			 (q0) [color=orange]  							{$q_0$};
		\node[state, thick]  			 			(q1) [right of=q0]   							   {$q_1$};
		\node[state, thick]     					(q2) [right of=q1]   							  {$q_2$};
		\node[state, thick]           				(q3) [right of=q2]   							   {$q_3$};
		\node[state, thick]           				(q4) [right of=q3]   							   {$q_4$};
		\node[state, thick, accepting]     (q5) [below of=q2, color=orange]   	{$q_5$};
		
		
		\path[->] 	 (q0) edge [thick, color=orange]      node [align=center]  {$e$} (q1)
							(q1) edge [thick]     							 node [align=center]  {$b$} (q2)
							(q1) edge [thick, bend right=-40 ]    node [align=center]  {$e$} (q3)
							(q2) edge [thick, bend right=-10 ]    node [align=center]  {$a$} (q3)
							(q3) edge [thick, bend right=-10 ]    node [align=center]  {$b$} (q2)
							(q3) edge [thick]     							   node [align=center]  {$a$} (q4)
							(q3) edge [thick,loop above]   		     node [align=center]  {$a$} (q3)
							(q4) edge [thick, bend right=-25]     node [align=center]  {$e$} (q1)
							(q2) edge [thick, color=orange]       node [align=center]  {$e$} (q5);
	\end{tikzpicture}
\end{center}

%%%%%%%%%%%%%%%%%%%%%%%%%%%%%%% Eliminate q4 %%%%%%%%%%%%%%%%%%%%%%%%%%%%%%%
\begin{center}
	\begin{tikzpicture}[>=stealth',shorten >=1pt,auto, node distance=2.3 cm, scale = 1, transform shape]
		\node[initial,state, thick]  			 (q0) [color=orange]  							{$q_0$};
		\node[state, thick]  			 			(q1) [right of=q0]   							   {$q_1$};
		\node[state, thick]     					(q2) [right of=q1]   							  {$q_2$};
		\node[state, thick]           				(q3) [right of=q2]   							   {$q_3$};
		\node[state, thick, accepting]     (q5) [below of=q2, color=orange]   	{$q_5$};
		
		
		\path[->] 	 (q0) edge [thick, color=orange]      						 node [align=center]  									   {$e$} (q1)
							(q1) edge [thick]     							 						node [align=center]  									  {$b$} (q2)
							(q1) edge [thick, bend right=-40]   					   node [align=center]  									 {$e$} (q3)
							(q2) edge [thick, bend right=-10]   					   node [align=center] 									 	 {$a$} (q3)
							(q3) edge [thick, bend right=-10]   					   node [align=center]  									 {$b$} (q2)
							(q3) edge [thick,loop above]   		   						  node [align=center]  										{$a$} (q3)
							(q3) edge [thick, color=red, , bend right=-37]     node [align=center, xshift=-10pt, above]  {$a$} (q1)
							(q2) edge [thick, color=orange]       						node [align=center]  									  {$e$} (q5);
	\end{tikzpicture}
\end{center}

%%%%%%%%%%%%%%%%%%%%%%%%%%%%%%% Eliminate q2 %%%%%%%%%%%%%%%%%%%%%%%%%%%%%%%
\begin{center}
	\begin{tikzpicture}[>=stealth',shorten >=1pt,auto, node distance=2.3 cm, scale = 1, transform shape]
		\node[initial,state, thick]  			 (q0) [color=orange]  																		 	 {$q_0$};
		\node[state, thick]  			 			(q1) [right of=q0]   							   												   {$q_1$};
		\node[state, thick]           				(q3) [right=3cm of q1]   					  												  {$q_3$};
		\node[state, thick, accepting]      (q5) [below right = 0.9cm and -2.5cm of q3, color=orange]   	{$q_5$};
		
		
		\path[->] 	 (q0) edge [thick, color=orange]      						 node [align=center]  						{$e$} (q1)
							(q1) edge [thick, color=red, below]        				  node [align=center, xshift=-5pt] 	{$b$} (q5)
							(q1) edge [thick, color=red, bend right=-15]   	   node [align=center]  					  {$ba$} (q3)
							(q3) edge [thick,loop above]   		   						  node [align=center]  						 {$a$\color{red}{$\cup ba$}} (q3)
							(q3) edge [thick, color=red]       							    node [align=center]  					  {$b$} (q5)
							(q3) edge [thick, bend right=-15]     					   node [align=center, above]  			{$a$} (q1);
	\end{tikzpicture}
\end{center}

%%%%%%%%%%%%%%%%%%%%%%%%%%%%%%% Eliminate q1 %%%%%%%%%%%%%%%%%%%%%%%%%%%%%%%
\begin{center}
	\begin{tikzpicture}[>=stealth',shorten >=1pt,auto, node distance=2.3 cm, scale = 1, transform shape]
		\node[initial,state, thick]  			 (q0) [color=orange]  							 												 {$q_0$};
		\node[state, thick]           				(q3) [right=3cm of q0]   					  												  {$q_3$};
		\node[state, thick, accepting]      (q5) [below right = 0.9cm and -2.5cm of q3, color=orange]   	{$q_5$};
		
		
		\path[->] 	 (q0) edge [thick, color=red]      node [align=center]  									   {$ba$} (q3)
							(q0) edge [thick, color=red]      node [align=center]  									  {$b$} (q5)
							(q3) edge [thick,loop above]    node [align=center]  										{$(a\cup ba)$\color{red}{$\cup( aba)$}} (q3)
							(q3) edge [thick]       				  node [align=center]  									 {$b$\color{red}{$\cup ba$}} (q5);
	\end{tikzpicture}
\end{center}


%%%%%%%%%%%%%%%%%%%%%%%%%%%%%%% Eliminate q3 %%%%%%%%%%%%%%%%%%%%%%%%%%%%%%%
\begin{center}
	\begin{tikzpicture}[>=stealth',shorten >=1pt,auto, node distance=2.3 cm, scale = 1, transform shape]
		\node[initial,state, thick]  			 (q0) [color=orange]  							 			{$q_0$};
		\node[state, thick, accepting]     (q5) [right = 7cm of q0, color=orange]   	{$q_5$};
		
		\path[->]  (q0) edge [thick, color=red]	node [align=center]		{$\Bigg( ba\bigg((a\cup ba)\cup( aba) \bigg)^*b\cup ba \Bigg) \bigcup b$} (q5);
	\end{tikzpicture}
\end{center}

\section{Κανονικές γλώσσες}
\subsection[Ερώτημα α]{\textbf{α) }} Η τομή μιας κανονικής γλώσσας με μια μη κανονική γλώσσα είναι πάντα κανονική γλώσσα.

\subsection[Ερώτημα β]{\textbf{β) }} Το συμπλήρωμα μιας πεπερασμένης γλώσσας είναι πάντα κανονική γλώσσα.

\subsection[Ερώτημα γ]{\textbf{γ) }} Κάθε μη κανονική γλώσσα περιέχει μη μετρήσιμο πλήθος συμβολοσειρών.

\section{Ελαχιστοποίηση καταστάσεων}

\begin{center}
	\begin{tikzpicture}[>=stealth',shorten >=1pt,auto,node distance=2.5 cm, scale = 1, transform shape]
		\node[initial,state, thick]  			 (q2)   						  {$q_2$};
		\node[state, thick]					        (q3) [right of=q2]     {$q_3$};
		\node[state, thick]           				(q1) [right of=q3]     {$q_1$};
		\node[state, thick]           				(q7) [right of=q1]     {$q_7$};
		\node[state, thick]							(q5) [below of=q2]  {$q_5$};
		\node[state, thick]							(q6) [below of=q3]  {$q_6$};
		\node[state, thick, accepting]	   (q4) [below of=q1]  {$q_4$};
		\node[state, thick]							(q8) [below of=q7] 	{$q_8$};
		
		\path[->] (q1) edge [thick]     							node [align=center]  												  {$a$} (q7)
						(q1) edge [thick, bend right=-10]     node [align=center]  													{$b$} (q8)
						(q2) edge [thick]    							   node [align=center]  												 {$a$} (q3)
						(q2) edge [thick, bend right=-10]     node [align=center] 		 											{$b$} (q5)
						(q3) edge [thick, loop right]     		   node [align=center]  												 {$a$} (q3)
						(q3) edge [thick, bend right=-10]     node [at start, below, xshift=10pt, yshift=-10pt]  {$b$} (q4)
						(q4) edge [thick]    							   node [align=center, below]									   {$a$} (q8)
						(q4) edge [thick]    							   node [align=center, right]  										 {$b$} (q1)
						(q5) edge [thick]    							   node [align=center]  										 		 {$a$} (q6)
						(q5) edge [thick, bend right=-10]     node [align=center]  													{$b$} (q2)
						(q6) edge [thick, loop right]     		   node [align=center]  												 {$a$} (q6)
						(q6) edge [thick, bend right=10]      node [at start, above, xshift=9pt, yshift=10pt]      {$b$} (q1)
						(q7) edge [thick, loop right]     		   node [align=center]  												 {$a$} (q7)
						(q7) edge [thick, bend right=-120]   node [align=center]  												   {$b$} (q4)
						(q8) edge [thick, bend right=-30]     node [align=center] 	 												{$a$} (q5);
	\end{tikzpicture}
\end{center}

\subsection[Ερώτημα α]{\textbf{α) }} Κατασκευάστε αναλυτικά το ισοδύναμο πρότυπο αυτόματο 
\subsection[Ερώτημα β]{\textbf{β) }} Πόσες κλάσεις ισοδυναμίας έχει κάθε μία από τις παρακάτω σχέσεις;
\subsection[Ερώτημα γ]{\textbf{γ) }} Περιγράψτε τις κλάσεις ισοδυναμίας της σχέσης 
