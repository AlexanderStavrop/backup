\section{Κανονικές Εκφράσεις}

\subsection[Ερώτημα α]{\textbf{α) }} L = \{w $\in \{a, b\}^*$ : η w περιέχει ακριβώς 2 εμφανίσεις του a και άρτιο αριθμό από b\}
\begin{equation*}
	L = \mathcal{L} \Bigg( (bb)^* \bigg( \Big( (ab \cup ba) (bb)^*(ab \cup ba) \Big)  \bigcup    \Big(aa\Big)  \bigg) (bb)^*\Bigg)
\end{equation*}

\subsection[Ερώτημα β]{\textbf{β) }} L = \{w $\in {a, b}^*$ : η w αρχίζει και τελειώνει με το ίδιο σύμβολο και έχει περιττό μήκος\}
\begin{equation*}
	L = \mathcal{L} \Bigg( \bigg(  a \Big( (a \cup b)(a \cup b) \Big)^* (a \cup b) a \bigg)  \bigcup   \bigg( b \Big( (a \cup b)(a \cup b) \Big)^* (a \cup b) b \bigg) \Bigg)
\end{equation*}

\subsection[Ερώτημα γ]{\textbf{γ) }} L = \{w $\in {a, b}^*$ : το πλήθος των a στην w είναι 4k + 1 (k $\geq$ 0) και δεν εμφανίζονται συνεχόμενα a\}
\begin{equation*}
	L = \mathcal{L} \bigg( b^*(a b^+a b^+a b^+a b^+)^*ab^* \bigg)
\end{equation*}




